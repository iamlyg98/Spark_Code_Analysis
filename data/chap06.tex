\chapter{Job运行和调度器模块}
\label{jobrunning}
RDD上的操作分为两种类型转换和动作。第\ref{chap:rddcreate}章中RDD的转换会构成DAG,最后RDD会触发action,WordCount例子中触发action的算子为saveAsTextFile。此算子会将数据集中的元素以textFile的形式保存到传入的路径中,本地文件或者HDFS。RDD中的saveAsTextFile经过一系列runJob的调用,最后调用了DagScheduler中的runJob。DagScheduler在第\ref{scinitial}章中已被初始化。其代码如程序\ref{inputPrg:dagrunjob}所示。
\begin{codeInput}{Scala}{DagScheduler.runJob}{dagrunjob}
def runJob[T, U](rdd: RDD[T],func: (TaskContext, Iterator[T]) => U,partitions: Seq[Int],callSite: CallSite,resultHandler: (Int, U) => Unit,properties: Properties): Unit = {
  ......
  val waiter = submitJob(rdd, func, partitions, callSite, resultHandler, properties)
  waiter.awaitResult() match {
    case JobSucceeded =>
      (waiter.jobId, callSite.shortForm, (System.nanoTime - start) / 1e9))
    case JobFailed(exception: Exception) =>
      (waiter.jobId, callSite.shortForm, (System.nanoTime - start) / 1e9))
      val callerStackTrace = Thread.currentThread().getStackTrace.tail
      exception.setStackTrace(exception.getStackTrace ++ callerStackTrace)
      throw exception
  }
}
\end{codeInput}

DagScheduler.runJobz主要调用了submitJob方法和waiter.awaitResult(),这里可以看出任务的运行是异步的。submitJob方法如程序\ref{inputPrg:submitjob}所示,代码中只保留了重要部分。
\begin{codeInput}{Scala}{DAGScheduler.submitJob}{submitjob}
def submitJob[T, U](
rdd: RDD[T],
func: (TaskContext, Iterator[T]) => U,
partitions: Seq[Int],
callSite: CallSite,
resultHandler: (Int, U) => Unit,
properties: Properties): JobWaiter[U] = {
......
val jobId = nextJobId.getAndIncrement()
......
val waiter = new JobWaiter(this, jobId, partitions.size, resultHandler)
eventProcessLoop.post(JobSubmitted(
  jobId, rdd, func2, partitions.toArray, callSite, waiter,
  SerializationUtils.clone(properties)))
  waiter
}
\end{codeInput}

由程序\ref{inputPrg:submitjob}可以看出submitJob会想生成一个jobId,并且生成jobwaiter来监听job的运行状态,最后通过DAGScheduler的事件处理线程进行处理。dag-scheduler-event-loop在SparkContex中初始化且作为守护线程存在。DAGSchedulerEventProcessLoop收到JobSubmitted事件后会立即调用DAGScheduler.handleJobSubmitted方法进行处理,这方法中会进行Stage的划分以及提交Stage。
\section{DagScheduler实现}
\subsection{Stage的划分}

在Spark中,一个Job会被分成多个Stage,各个直接存在着依赖关系,其中最下游的Stage也成为finalStage或resultStage。并行在不同分区进行相同的逻辑计算且互不影响的任务属于同一个Stage,而shuffle过程要从父RDD的多个分区中拉取数据,必须要等待多个分区任务完成才能开始计算,故其不能并行化,属于不同的Stage,DAG中划分Stage就是以Shuflle为依据进行的。

handleJobSubmitted通过调用org.apache.spark.DAGScheduler\#newResultStage来创建finalStage,
\begin{orgianlboxnotitle}
	finalStage = newResultStage(finalRDD, func, partitions, jobId, callSite)
\end{orgianlboxnotitle}

newResultStage会根据传入的finalRDD来获取父Stage和jobId,然后开始创建ResultStage即当前的Stage,这里父RDD使用列表类型。代码如程序\ref{inputPrg:newresultstage}所示
\begin{codeInput}{Scala}{DAGScheduler.newResultStage}{newresultstage}
private def newResultStage(rdd: RDD[_],func: (TaskContext, Iterator[_]) => _,partitions: Array[Int],
jobId: Int,callSite: CallSite): ResultStage = {
  val (parentStages: List[Stage], id: Int) = getParentStagesAndId(rdd, jobId)
  val stage = new ResultStage(id, rdd, func, partitions, parentStages, jobId, callSite)
  stageIdToStage(id) = stage
  updateJobIdStageIdMaps(jobId, stage)
  stage
}
\end{codeInput}

程序\ref{inputPrg:newresultstage}中getParentStagesAndId方法会进一步调用getParentStages方法,getParentStages方法为划分Stage的核心。它的返回类型为List[Stage],每遇到DAG中RDD的ShuffleDependency就会参数一个新的Stage,其代码如程序\ref{inputPrg:getparentstages}所示。
\begin{codeInput}{Scala}{DAGScheduler.getParentStages}{getparentstages}
private def getParentStages(rdd: RDD[_], firstJobId: Int): List[Stage] = {
  val parents = new HashSet[Stage]
  val visited = new HashSet[RDD[_]]
  val waitingForVisit = new Stack[RDD[_]]
  def visit(r: RDD[_]) {
    if (!visited(r)) {
      visited += r
      for (dep <- r.dependencies) {
        dep match {
          case shufDep: ShuffleDependency[_, _, _] =>
            parents += getShuffleMapStage(shufDep, firstJobId)
          case _ =>waitingForVisit.push(dep.rdd)
        }
      }
    }
  }
  waitingForVisit.push(rdd)
  while (waitingForVisit.nonEmpty) {
    visit(waitingForVisit.pop())
  }
  parents.toList
}
\end{codeInput}

getParentStages方法中定义了一个visit方法,此方法会对方法外的变量parents、visited及waitingForVisit产生影响,这里用到Scala函数的闭包特性。getParentStages变量说明如下
\begin{centertitlebox}{getParentStages变量说明}
	parents:存放各个ShuffleDependency所依赖的Stage\\
	visited:存放DAG中已经遍历过的RDD\\
	waitingForVisit:存放即将要遍历的RDD,为栈结构
\end{centertitlebox}
\subsection{实例:WordCount划分Stage}

以本WordCount样例展示如何划分Stage。WordCount DAG如图\ref{fig:SparkCountDetails}所示,提前三个重要的信息组成新的RDD数据结构,定义如下
\begin{centertitlebox}{新定义RDD数据结构}
	数据结构:RDD(deps:List[Dependency],partitions:List(partitionId))\\
	deps:代表当前RDD的依赖列表\\
	partitions:指当前RDD的分区列表\\
	textFile:\\  HadoopRDD[1](Nil,List(1,2))\\
	flatMap:\\  MapPartitionRDD[2](List(onoToOneDependency(HadoopRDD[1]),List(1,2))\\
	map:\\  MapPartitionRDD[3](List(onoToOneDependency(MapPartitionRDD[2])),List(1,2))\\
	reduceByKey:\\  ShuffledRDD[4](List(ShuffleDependency(MapPartitionRDD[3])),List(3,4))\\
	filter:\\  MapPartitionRDD[5](List(onoToOneDependency(ShuffledRDD[4])),List(3,4))\\
	saveTextFile:\\  MapPartitionRDD[6](List(onoToOneDependency(MapPartitionRDD[5])),List(3,4))
\end{centertitlebox}

这里定义RDD中包含了两种数据元素,分别为当前RDD的依赖列表和当前RDD的分区列表。依赖列表中存放有当前RDD所依赖的父RDD的对象,分区列表中存放着分区的Id,要获取该分区数据可以通过BlockManager来获取。

DAG以finalRDD即MapPartitionRDD[6]为起始点,依次进入函数getParentStages,各变量参数的值变化情况如表\ref{tab:getstage}所示
\begin{table}[H]
	\caption{getParentStages变量值}
	\label{tab:getstage}
	\begin{tabularx}{\linewidth}{lXX}
		\toprule[1.5pt]
		{\heiti visited:HashSet} & {\heiti parents:HashSet} &{\heiti waittingForVisit:Stack}\\
		\midrule[1pt]
		6& Nil &6 \\
		6& Nil &5 \\
		6 5& Nil &4 \\
		6 5 4& ShuffleMapStage(3,shuffleDep(3)) &Nil \\
		\bottomrule[1.5pt]
	\end{tabularx}
\end{table}

这里函数生成了ShuffleMapStage,其中存放了RDD[3]的信息,另外ShuffleMapStage中Task计算的结果会传给Driver端的mapOutputTracker,其他任Task可以通过查询它来获取这些结果,不过这些结果并不是真实的数据,而是保存真实数据所在位置、大小等元数据信息,其他Task通过这些元数据信息获取其需要处理的数据。接下来回到程序\ref{inputPrg:newresultstage},进入newResultStage代码块。最后返回ResultStage(RDD[6],List(ShuffleMapStage(3,shuffleDep(3))))。最后函数newResultStage将jobId和ResultStage进行绑定,返回ResultStage,也就是finalStage。
\section{任务调度器实现}
\subsection{任务的生成}

现在回到handleJobSubmitted,上节生成了finalStage,此方法后面部分会调用submitStage来提交这个Stage,如果当前提交的Stage还有父Stage没有提交,那么就递归提交父Stage,只有父Stage提交完了,才能提交当前Stage,submitStage如程序\ref{inputPrg:submitStage}所示
\begin{codeInput}{Scala}{DAGScheduler.submitStage}{submitStage}
private def submitStage(stage: Stage) {
  val jobId = activeJobForStage(stage)
  if (jobId.isDefined) {
    if (!waitingStages(stage) && !runningStages(stage) && !failedStages(stage)) {
      val missing = getMissingParentStages(stage).sortBy(_.id) //List(ShuffleMapStage[0])
      if (missing.isEmpty) {
        submitMissingTasks(stage, jobId.get)
      } else {
        for (parent <- missing) {
          submitStage(parent)
        }
        waitingStages += stage//此时waitingStages=set(ResultStage 1)
      }
    }
  } else {
    abortStage(stage, "No active job for stage " + stage.id, None)
  }
}
\end{codeInput}

经过上节DAGScheduler对DAG Stage的划分,WordCount实例逻辑上可以形成图\ref{fig:SparkCountDetails}所示的Stage。Stage1作为submitStage的参数进入,函数先判断其是否有父Stage,有父Stage就递归提交父Stage,最后会由submitMissingTasks来完成最后的提交任务工作。submitMissingTasks函数会首先分析其包含的分区数,对于每一个分区会为其生成一个Task,然后同属于一个Stage的Task会被封装为TaskSet,最后提交给TaskScheduler。TaskSet中包含了一组处理逻辑完全相同的Task,只是作用数据不同。
\subsection{TaskScheduler的创建}

对于每个TaskScheduler都会对应一个SchedulerBackend。其中TaskScheduler负责程序的不同job之间的调度,SchedulerBackend负责与集群资源管理器进行通信,取得应用所需要的资源,并且将资源传递给TaskScheduler,由TaskScheduler为其最后分配计算资源。

TaskScheduler和SchedulerBackend在SparkContext初始化的时候创建,不同的资源管理模式和部署模式下其对应的值不相同,在SparkContext初始化章节中分析了Yarn-Cluster模式下TaskScheduler和SchedulerBackend分别对应YarnClusterScheduler和YarnClusterSchedulerBackend。这里TaskScheduler和SchedulerBackend都是接口,SchedulerBackend负责分配当前可用的资源,具体就是向当前等待分配计算资源的Task分配计算资源。

\subsection{Task的提交}

submitMissingTasks中会通过TaskScheduler.submitTasks来提交Tasks,TaskScheduler提供的是一个接口,在不同模式下会有不同实现。在Yarn-Cluster模式下会调用TaskSchedulerImpl.submitTasks(TaskSet),其作用首先将保存这组任务的TaskSet加入到一个TaskManager中,TaskManager会根据数据的locality aware为Task分配资源,并且也对Task的执行状态监控。接着为应用分配调度策略,主要包括FIFO和FAIR。默认情况下为FIFO,调度的单位为TaskSet,优先调度jobId小的以及先到rootPool的TaskSet。最后将调用SchedulerBackend.reviveOffers(),实际调用CoarseGrainedSchedulerBackend.reviveOffers(),接着调用CoarseGrainedSchedulerBackend.DriverEndpoint.makeOffers(),遍历出每个Executor的资源封装之后,传入TaskSchlerImpl.resourceOffers()为每个Task分配具体的Executor,分配资源部分代码如程序\ref{inputPrg:allocateRes}所示
\begin{codeInput}{Scala}{TaskSchedulerImpl.resourceOffers}{allocateRes}
//为了避免Task集中分配到某些机器上,随机打乱
val shuffledOffers = Random.shuffle(offers)
//存储分配好的Task
val tasks = shuffledOffers.map(o => new ArrayBuffer[TaskDescription](o.cores))
val availableCpus = shuffledOffers.map(o => o.cores).toArray
val sortedTaskSets = rootPool.getSortedTaskSetQueue
for (taskSet <- sortedTaskSets) {
  taskSet.parent.name, taskSet.name, taskSet.runningTasks))
  if (newExecAvail) {
    taskSet.executorAdded()
  }
}
var launchedTask = false
for (taskSet <- sortedTaskSets; maxLocality <- taskSet.myLocalityLevels) {
  do {
    launchedTask = resourceOfferSingleTaskSet(taskSet, maxLocality, shuffledOffers, availableCpus, tasks)
  } while (launchedTask)
}
if (tasks.size > 0) {
  hasLaunchedTask = true
}
return tasks
\end{codeInput}

程序返回的是TaskDescription的二维数组,TaskDescription包含的是TaskId、ExecutorId和Task执行所需要的依赖环境信息等。

最后调用CoarseGrainedSchedulerBackend.DriverEndPoint.launchTasks,它的输入为一个TaskDescription的二维数组,即上面程序段的返回值。此函数的作用先对TaskDescription进行序列化,然后更新Executor资源状况,空闲cpu core减去每个Task需要的cpu core,最后将Task通过Rpc发送到Executor,Executor收到这个消息后就会开始执行Task。相应的代码如程序\ref{inputPrg:driverlaunchtasks}所示
\begin{codeInput}{Scala}{CoarseGrainedSchedulerBackend.DriverEndPoint.launchTasks}{driverlaunchtasks}
private def launchTasks(tasks: Seq[Seq[TaskDescription]]) {
  for (task <- tasks.flatten) {
  val serializedTask = ser.serialize(task)
  if (serializedTask.limit >= akkaFrameSize - AkkaUtils.reservedSizeBytes) {
    scheduler.taskIdToTaskSetManager.get(task.taskId).foreach { taskSetMgr =>
    try {
      taskSetMgr.abort(msg)
    } catch {
      case e: Exception => logError("Exception in error callback", e)
    }
  }
  }
  else {
    val executorData = executorDataMap(task.executorId)
    executorData.freeCores -= scheduler.CPUS_PER_TASK
    executorData.executorEndpoint.send(LaunchTask(new SerializableBuffer(serializedTask)))
    }
  }
}
\end{codeInput}

Executor执行Task会在下一章中进行介绍。
\subsection{Task计算结果的处理}
\label{driverdealresult}
Executor在执行完Task时会向Driver发送StatusUpdate的消息来通知Driver任务的状态更新未TaskState.FINISHED。Driver会通知TaskScheduler,TaskScheduler会执行TaskSchedulerImpl.statusUpdate,其代码如程序\ref{inputPrg:taskimplstatus}所示
\begin{codeInput}{Scala}{TaskSchedulerImpl.statusUpdate}{taskimplstatus}
taskIdToTaskSetManager.get(tid) match {
  case Some(taskSet) => 
    if (TaskState.isFinished(state)) {
    taskIdToTaskSetManager.remove(tid)
    taskIdToExecutorId.remove(tid).foreach { execId =>
      if (executorIdToTaskCount.contains(execId)) {
        executorIdToTaskCount(execId) -= 1
     }
    }
  }
  if (state == TaskState.FINISHED) {
    taskSet.removeRunningTask(tid)
    taskResultGetter.enqueueSuccessfulTask(taskSet, tid, serializedData)
  } else if (Set(TaskState.FAILED, TaskState.KILLED, TaskState.LOST).contains(state)) {
    taskSet.removeRunningTask(tid)
    taskResultGetter.enqueueFailedTask(taskSet, tid, state, serializedData)
  }
  case None =>
    logError(
    ("Ignoring update with state %s for TID %s because its task set is gone (this is " +
    "likely the result of receiving duplicate task finished status updates)")
    .format(state, tid))
}
\end{codeInput}

Task只有在FINISHED状态下,才会被标记为成功,其他状态为执行失败。对于成功的Task,Driver会将Task加入到成功任务队列。Executor执行Task成功返回的结果可能有两种情形,
\begin{enumerate}[\bfseries 1]
    \item 直接结果
    
    这种情况下直接返回结果即可
    \item 非直接结果
    
    需要向远程worker网络获取结果,获取之后在远程节点上删除结果,这通过Spark中的BlockManager完成。
\end{enumerate}

这两种结果的产生会在下章Executor执行Task章节讲述。

接着通过TaskScheduleImpl.handleSuccessfulTask来负责处理获取到的计算结果。这里调用栈如下所示
\begin{centertitlebox}{handleSuccessfulTask调用栈}
	(1)TaskScheduleImpl.handleSuccessfulTask\\
	(2)TaskSetManager.handleSuccessfulTask\\
	(3)DAGScheduler.taskEnded\\
	(4)DAGScheduler.handleTaskCompletion
\end{centertitlebox}

其中最核心的为第4个,它会对不同的Task进行模式匹配,进行不同的处理,对于ShuffleMapTask,程序代码如程序\ref{inputPrg:shufflemaptask}所示
\begin{codeInput}{Scala}{ShuffleMapTask结果处理}{shufflemaptask}
case smt: ShuffleMapTask =>
  val shuffleStage = stage.asInstanceOf[ShuffleMapStage]
  updateAccumulators(event)
  val status = event.result.asInstanceOf[MapStatus]
  val execId = status.location.executorId
  if (failedEpoch.contains(execId) && smt.epoch <= failedEpoch(execId)) {
  } else {
    shuffleStage.addOutputLoc(smt.partitionId, status)
  }
  if (runningStages.contains(shuffleStage) && shuffleStage.pendingPartitions.isEmpty) {
    markStageAsFinished(shuffleStage)
    mapOutputTracker.registerMapOutputs(
    shuffleStage.shuffleDep.shuffleId,
    shuffleStage.outputLocInMapOutputTrackerFormat(),
    changeEpoch = true)	
    clearCacheLocs()	
    if (!shuffleStage.isAvailable) {
      shuffleStage.findMissingPartitions().mkString(", "))
      submitStage(shuffleStage)
    } else {
      if (shuffleStage.mapStageJobs.nonEmpty) {
        val stats = mapOutputTracker.getStatistics(shuffleStage.shuffleDep)
        for (job <- shuffleStage.mapStageJobs) {
          markMapStageJobAsFinished(job, stats)
        }
      }
    }
}
\end{codeInput}

首先会将整体结果注册到MapOutputTracker中去,这样下一个Stage的Task就可以方便的获取所需数据的元数据信息。接着判断当前Stage中是否所有任务都已成功返回,如果有部分数据是空的,那证明执行该分区的Task执行失败,需要重新提交Stage;如果当前Stage执行成功并且没有父Stage,那么就通过submitWaitingStages()将waitingStage集合中的Stage提交。

至此一个Stage就已经执行完成,如果Stage是ResultStage,那么一个job就执行完成了。

Spark中有两种Task,ShuffleMapTask和ResultTask。ShuffleMapTask根据Task的Partition存放计算结果,而ResultTask将计算结果发送到Driver。用户程序出发action后,会调用SparkContext的runJob方法开始进行任务的提交。最后会通过DAG的事件处理器传递到DAGScheduler.handleJobSubmitted,它会首先划分Stage,然后提交Task。至此,Task开始在集群上运行了。
Stage的开始就是从外部存储或者shuffle结果中读取数据;一个Stage的结束就是由于发生shuflle或者生成结果。
